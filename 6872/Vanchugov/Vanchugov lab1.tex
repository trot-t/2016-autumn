\documentclass{article}
\usepackage[utf8]{inputenc}
\usepackage[russian]{babel}

\title{Scilab. Лабораторная 2}
\author{Ванчугов Е.О }
\begin{document}
\maketitle
{\Large Задание 1}: Найти ребро куба, равновеликого шару, площадь поверхности которого равна площади боковой поверхности прямого кругового конуса, у которого высота вдвое меньше, чем длина образующей. Объем этого конуса равен 1.

Объем конуса $V_k = 1$

Найдем радиус основания конуса r:
$$r=\sqrt{3}{\frac{V_k*\sqrt3}{\pi}} = 0.8199806 $$

Вычислим длину образующей:
$$l=\frac{r*2}{\sqrt3} = 0.9468321 $$

Формула поверхности конуса:
$$s=\pi*r*l = 2.4390821 $$

Найдем радиус шара:
$$R=\sqrt{\frac{s}{4*\pi}} = 0.4405633$$

Вычислим объем шара:
$$V_{sh}=\frac{4*\pi*R^3}{3} = 0.3581900$$

Заключительная формула для нахождения ребра куба:
$$a=\sqrt[3]{V_{sh}} = 0.7101844$$

Исходный код для Scilab
\begin{verbatim}
Vk=1
r=nthroot((Vk*sqrt(3))/%pi, 3)
l=(r*2)/sqrt(3)
s=%pi*r*l
R=sqrt(s/(4*%pi))
Vsh=(4*%pi*R^3)/3
a=nthroot(Vsh,3)
\end{verbatim}

\end{document}
