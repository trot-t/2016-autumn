\documentclass[a4paper,11pt]{article}
\usepackage[T2A]{fontenc}
\usepackage[utf8]{inputenc}
\usepackage[english,russian]{babel}
\title{practic1}
\author{Нагорных.В гр.6872}
%\date{September 2016}

\begin{document}
\maketitle

Для перевода двоичного числа в десятичное необходимо его записать в виде многочлена,
состоящего из произведений цифр числа и соответствующей степени числа два, и вычислить


$
 10101_2=1\cdot2^{4}+0\cdot2^{3}+1\cdot2^{2}+0\cdot2^{1}+1\cdot2^{0}=21_10
$


Для перевода десятичного числа в двоичное его необходимо последовательно делить на два до тех пор,
пока не осттанется остаток, меньший или равный еденице.Число в двоичной системе записывается как
последовательность последнего результата деления и остатков в обратном порядке

$
\frac{21}{2}=10  \textcyrillic { остаток } 1 
$

$ 
\frac{10}{2}=5  \textcyrillic { остаток } 0
$

$
\frac{5}{2}=2 \textcyrillic { остаток } 1 
$

$ 
\frac{2}{2}=1 \textcyrillic  { остаток } 0
$


\end{document}
